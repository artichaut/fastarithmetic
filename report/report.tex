\documentclass[a4paper,11pt]{article}
\usepackage[utf8]{inputenc}
\usepackage[T1]{fontenc}
\usepackage[english]{babel}
\usepackage{amsmath,amssymb,amsthm,amsopn}
\usepackage{mathrsfs}
\usepackage{graphicx}
\usepackage{hyperref}
%\usepackage{tikz}
\usepackage{array}
%\usepackage[top=1cm,bottom=1cm]{geometry}
%\usepackage{listings}
%\usepackage{xcolor}

% Création des labels Théorème, Lemme, etc...

\newtheoremstyle{break}%
{}{}%
{\itshape}{}%
{\bfseries}{}%  % Note that final punctuation is omitted.
{\newline}{}

\theoremstyle{break}
\newtheorem{thm}{Théorème}[section]
\newtheorem{lm}[thm]{Lemme}
\newtheorem{prop}[thm]{Proposition}
\newtheorem{cor}[thm]{Corollaire}

\theoremstyle{definition}
\newtheorem{defi}[thm]{Définition}
\newtheorem{ex}[thm]{Exemple}

\theoremstyle{remark}
\newtheorem{rem}[thm]{Remarque}

% Raccourcis pour les opérateurs mathématiques (les espaces avant-après sont
% modifiés pour mieux rentrer dans les codes mathématiques usuels)
% \deg already exists !!
\DeclareMathOperator{\Ker}{Ker}
\DeclareMathOperator{\Id}{Id}
\DeclareMathOperator{\Img}{Im}
\DeclareMathOperator{\Card}{Card}
\DeclareMathOperator{\Vect}{Vect}



% Nouvelles commandes
\newcommand{\ps}[2]{\left\langle#1,#2\right\rangle}
\newcommand{\ent}[2]{[\![#1,#2]\!]}

% opening
\title{Internship report}
\author{Édouard \textsc{Rousseau}\\Internship supervised by Luca \textsc{De Feo}}



\begin{document}

\maketitle

\begin{abstract}

  This internship took place during July and August 2016, the aim was to study
  the algorithms presented in \cite{DDS14}, and to implement them in Nemo
  \cite{Nemo}, a computer algebra package for the Julia \cite{Julia} programming
  language. This paper intends to compare the Julia and the C implementations of
  the code, in terms of speed, but also genericity and easyness to write and
  read such code.

\end{abstract}

\tableofcontents

\clearpage

\section{Julia}
\subsection{Overview of Julia's caracteristics}
Julia is a free and open-source, high-level programming language developed since 2012, with dynamic
type system and high-performance. It is a compiled language, with a
\emph{just-in-time} (jit) compilation. It means that when you write a
function, e.g. 
\begin{verbatim}
julia> function myFunction(n)
       return n^3+n+2
       end
\end{verbatim}
your function will be compiled during your very first call. After this first
call, it will be much faster to run the compiled function
\begin{verbatim}
julia> myFunction(BigInt(2)^300)
\end{verbatim}
than the expression below.
\begin{verbatim}
julia> (BigInt(2)^300)^3 + BigInt(2)^300 + 2
\end{verbatim}
In order to optimize the code, the compiler creates a new function for each
type of input that you can think of when calling myFunction. For example, in the
code 
below, Julia will compile myFunction three times, which will result in
three functions, each one of them optimized for \textbf{Int64} (the default type
of $17$), \textbf{Float64}, or \textbf{BigInt}.
\begin{verbatim}
julia> myFunction(17)
4932
julia> myFunction(17.)
4932.0
julia> myFunction(BigInt(17))
4932
\end{verbatim}
Julia is a very easy-to-learn language, I had never seen Julia code before the 
internship, and I am not an expert in computer science, but the writing part of
the code was not the hardest. 
The code written in FastArithmetic.jl is
very generic, thanks to the type system of Julia, and works for fields
$\mathbb{F}_{p^n}$ with a small $p$ as well as large $p$. More importantly, the
\emph{jit} compiler creates a function for each kind of $p$. Indeed the elements
of $\mathbb{F}_{p^n}$ have type \textbf{fq\_nmod} for $p$ small or \textbf{fq}
for $p$ large, so one can write only one function and have two compiled
functions : one for each type, which will be optimized by the compiler for this type.

\subsection{Nemo}

Nemo is a computer algebra package of Julia, you can install it with 
\begin{verbatim}
julia> using Nemo
\end{verbatim}
and test it with the following.
\begin{verbatim}
julia> Pkg.test(``Nemo'')
\end{verbatim}
Nemo is based on C/C++ libraries such as Flint, Antic, Pari, etc. and is also
written in Julia. It provides a lot of features, and all the work of this
internship is based on Nemo. Indeed, we work with Nemo finite fields and Nemo
polynomials over these fields. \textbf{fq}, \textbf{fq\_nmod}, and
\textbf{fq\_nmod\_poly} (the type of the polynomials over $\mathbb{F}_{p^n}$) are
all Nemo types. You can learn more in Nemo's manual, which is very well
documented.

\section{A bit of theory}

\section{Julia/Nemo in practice}
Since Julia is easy-to-write, I hope it's also
easy-to-read, if not, I think that it is due to my lack of experience in
software programing, rather than to Julia itself. You can find the code
at
https://gitbub.com/edouardRousseau/FastArithmetic.jl. What's more, it is really
easy to install personal packages, you just have to clone my repository
inside Julia, using
\begin{verbatim}
julia> Pkg.clone(``https://github/edouardRousseau/FastArithmetic.jl'') 
\end{verbatim}
and you can also test the package easily by running the following.
\begin{verbatim}
julia> Pkg.test(``FastArithmetic'') 
\end{verbatim}
You may want to update the package, because the code is very suceptible to change, so just use
\begin{verbatim}
julia> Pkg.update()
\end{verbatim}
and you will be ready to start again. 
\subsection{Benchmarks}
We will now look at the speed of the code in Julia, and compare it with
the speed in C. All the benchmarks in Julia were realised with the
@benchmark macro available in the BenchmarkTools package. This macro lets you
benchmark any function with 

\begin{verbatim}
julia>@benchmark 1+1
BenchmarkTools.Trial: 
  samples:          10000
  evals/sample:     1000
  time tolerance:   5.00%
  memory tolerance: 1.00%
  memory estimate:  0.00 bytes
  allocs estimate:  0
  minimum time:     3.00 ns (0.00% GC)
  median time:      3.00 ns (0.00% GC)
  mean time:        3.04 ns (0.00% GC)
  maximum time:     10.00 ns (0.00% GC)
\end{verbatim}
which runs the desired function and measures the time it takes. It's possible
to choose all the parameters of the benchmark and you can use different
estimators. For the following benchmarks, we use the median time, because it
is more stable than the mean time and still give an idea of the average time
the function will take. The stability we are talking about is linked with the
memory management in Julia. Indeed, in Julia, we don't have to concern about
memory, and the variables we assign are cleared by the Garbage Collector
(GC). The time Julia pass in the GC is not very clear nor predictible, and the
operations in the GC take a significant time. That's why the time taken by a
function can change between two samples, depending on the time spent in the GC.
By default, before every benchmark, the garbage if automaticaly collected, but
there is still a bit of unstabiliy.
Since all the functions written in FastArithmetic.jl depends on at least one
polynomial $P$, sometimes also $Q$ and $R=P\odot Q$, we let $m=\deg (P)$ grow
from $1$ to $200$ and benchmark the functions for all these values of $m$. We
choose $\deg (Q)=m+1$, and both $P$ and $Q$ are irreducible polynomials.
Irreducible polynomials of degree $1$ to $201$ were precomputed, by chosing
random elements in $\mathbb{F}_{p}[X]$, in order to
always use the same list and to be sure that we are not using special
polynomials that could be speeding up the functions (like sparse
polynomials). In the arrays, the speed of
the C code is set to $1.00$, and smaller is better. The speed of 
\textbf{monomialToDual} and of \textbf{dualToMonomial} are the same, either in C
or Julia, so we show the results only for the first.

\begin{center}
\begin{tabular}[here]{ccc}
\textbf{monomialToDual} & \textbf{mulT} & \textbf{embed} \\

  \begin{tabular}[here]{cc}
   $m$ & Julia \\
   $10$ & $5.12$\\
   $51$ & $1.14$\\
   $104$ & $0.56$\\
   $140$ & $0.44$\\
   $187$ & $0.38$\\

  \end{tabular}

  &
  
  \begin{tabular}[here]{cc}
   $m$ & Julia \\
   $10$ & $1.15$\\
   $51$ & $0.08$\\
   $104$ & $0.03$\\
   $140$ & $0.02$\\
   $187$ & $0.01$\\

  \end{tabular}

  &

  \begin{tabular}[here]{cc}
   $m$ & Julia \\
   $10$ & $31.30$\\
   $51$ & $37.17$\\
   $104$ & $35.29$\\
   $140$ & $33.87$\\
   $187$ & $30.87$\\

  \end{tabular}\\

    \textbf{project} & \textbf{phi1} & \textbf{phi2} \\

  \begin{tabular}[here]{cc}
   $m$ & Julia \\
   $10$ & $32.77$\\
   $51$ & $30.64$\\
   $104$ & $26.64$\\
   $140$ & $24.24$\\
   $187$ & $20.64$\\

  \end{tabular}

  &
 
  \begin{tabular}[here]{cc}
   $m$ & Julia \\
   $10$ & $25.83$\\
   $51$ & $29.55$\\
   $104$ & $22.53$\\
   $140$ & $22.94$\\
   $187$ & $22.12$\\

  \end{tabular}

  &
 
  \begin{tabular}[here]{cc}
   $m$ & Julia \\
   $10$ & $2.56$\\
   $51$ & $102$\\
   $104$ & $187$\\
   $140$ & $309$\\
   $187$ & $449$\\

  \end{tabular}
\end{tabular}
\end{center}


\clearpage
\bibliographystyle{unsrt}
\bibliography{biblio}
\end{document}
