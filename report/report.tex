\documentclass[a4paper,11pt]{article}
\usepackage[utf8]{inputenc}
\usepackage[T1]{fontenc}
\usepackage[english]{babel}
\usepackage{amsmath,amssymb,amsthm,amsopn}
\usepackage{mathrsfs}
\usepackage{graphicx}
\usepackage{hyperref}
%\usepackage{tikz}
%\usepackage{array}
%\usepackage[top=1cm,bottom=1cm]{geometry}
%\usepackage{listings}
%\usepackage{xcolor}

% Création des labels Théorème, Lemme, etc...

\newtheoremstyle{break}%
{}{}%
{\itshape}{}%
{\bfseries}{}%  % Note that final punctuation is omitted.
{\newline}{}

\theoremstyle{break}
\newtheorem{thm}{Théorème}[section]
\newtheorem{lm}[thm]{Lemme}
\newtheorem{prop}[thm]{Proposition}
\newtheorem{cor}[thm]{Corollaire}

\theoremstyle{definition}
\newtheorem{defi}[thm]{Définition}
\newtheorem{ex}[thm]{Exemple}

\theoremstyle{remark}
\newtheorem{rem}[thm]{Remarque}

% Raccourcis pour les opérateurs mathématiques (les espaces avant-après sont modifiés pour mieux rentrer dans les codes mathématiques usuels)
\DeclareMathOperator{\Ker}{Ker}
\DeclareMathOperator{\Id}{Id}
\DeclareMathOperator{\Img}{Im}
\DeclareMathOperator{\Card}{Card}
\DeclareMathOperator{\Vect}{Vect}


% Nouvelles commandes
\newcommand{\ps}[2]{\left\langle#1,#2\right\rangle}
\newcommand{\ent}[2]{[\![#1,#2]\!]}

% opening
\title{Internship report}
\author{Édouard \textsc{Rousseau}\\Internship supervised by Luca \textsc{De Feo}}



\begin{document}

\maketitle

\begin{abstract}

  This internship took place during July and August 2016, the aim was to study the algorithms presented in \cite{DDS14}, and to implement them in Nemo \cite{Nemo}, a computer algebra package for the Julia \cite{Julia} programming language. This paper intends to compare the Julia and the C implementations of the code, in terms of speed, but also genericity and easyness to write and read such code.

\end{abstract}

\tableofcontents

\clearpage

\section{Julia}
\subsection{Overview of Julia's caracteristics}
\subsection{Nemo}

\section{A bit of theory}

\section{Julia/Nemo in practice}
Julia is a very easy to learn language, I had never seen Julia code before the internship, and I am not an expert in computer science, but the writing part of the code was not the hardest at all. Since it's easy to write, I hope it's also easy to read, if not, I think that it is due to my lack of experience in software programing, rather than to Julia itself. You can find the code at https://gitbub.com/edouardRousseau/FastArithmetic.jl. What's more, it is really easy to install personal packages too, you just have to clone my repository inside Julia, using
\begin{verbatim}
julia> Pkg.clone(``https://github/edouardRousseau/FastArithmetic.jl'') 
\end{verbatim}
and you can also test the package easily by running the following.
\begin{verbatim}
julia> Pkg.test(``FastArithmetic'') 
\end{verbatim}
You may want to update the package, because the code is very suceptible to change, so just use
\begin{verbatim}
julia> Pkg.update()
\end{verbatim}
and you will be ready to start again. The code written in FastArithmetic.jl is very generic, thanks to the type system of Julia, and works for fields $\mathbb{F}_{p^n}$ with a small $p$ but for large $p$ as well. More importantly, the \emph{jit} compiler creates a function for each type of $p$ (\textbf{fq\_nmod} for $p$ small or \textbf{fq} for $p$ large), so one can write only one function and have two compiled functions : one for each type, which will be optimized by the compiler for this type.

\clearpage
\bibliographystyle{unsrt}
\bibliography{biblio}
\end{document}
